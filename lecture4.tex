\chapter{Memory Management}
\section{Pointers to pointers}
You can mutate a pointer argument by using a pointer to a pointer.
\begin{minted}{c}
void next_el(nt **h) {
	*h = *h + 1;
}

int main(void) {
	int A[] = { 10, 20, 30 };
	
	// *p == 10
	int *p = A;
	
	// mutates p
	next_el(&p);
	
	// now *p == 20
}
\end{minted}
\section{Strings in C}
C strings are null-terminated character arrays.
\begin{minted}{c}
int slen(char s[]) {
	int n = 0;
	while (s[n] != 0)
		n++;
	return n;
}

int main(void) {
	char str[] = "abc";
	
	slen(str);  // evaluates to 3.
}
\end{minted}

\subsection{Concise \texttt{strlen()}}
\begin{minted}{c}
int strlen(char *s) {
	char *p = s;
	while (*p++);  // go one past the end
	return p - s - 1;
}
\end{minted}

\subsection{Arguments to \texttt{main}}
\begin{minted}{c}
int main(int argc, char * argv[]);
\end{minted}
\begin{itemize}
	\item \texttt{argc} is the \emph{number} of commands
	\item \texttt{argv} points to an array of strings.
		\begin{itemize}
			\item pointer to an array of pointers!
			\item \texttt{argv[0]} is the name of the executable.
		\end{itemize}
\end{itemize}

\section{C memory management}
\begin{tabular}{ll}
	Address & Contents\\\hline
	FFFF\,FFFFF & stack (grows down)\\
	\(\vdots\) & \(\vdots\) \\
	? & heap (grows up) \\
	? & static data (mutable) \\
	0000\,0000 & code (immutable)
\end{tabular}
\begin{description}
	\item[code] does not change.
	\item[static data] loaded when program starts, \emph{can} be modified.
	\item[stack] local args, allocated when function is called, grows downward (from higher to lower addresses).
		\begin{itemize}
			\item \emph{stack pointer} is a register that moves down with deepening calls to track where the last frame is
		\end{itemize}
	\item[heap] space for dynamic data, allocated and freed as needed.
\end{description}

\section{Stack}
\begin{itemize}
	\item Contents of a frame:
		\begin{itemize}
			\item function arguments
			\item local variables
			\item who called me? (for a return)
		\end{itemize}
\end{itemize}

\section{Heap}
\begin{itemize}
	\item \texttt{malloc} allocates a block of memory
	\item \texttt{calloc} allocates a zeroed block of memory
	\item \texttt{free} frees previously allocated block of memory
	\item \texttt{realloc} reallocates for the same data but a different size
\end{itemize}

\subsection{\texttt{malloc()}}
\begin{minted}{c}
void * malloc(size_t n);
\end{minted}
\begin{itemize}
	\item \texttt{n} is an integer
	\item \texttt{size\_t} is an unsigned int that \emph{counts} memory bytes
	\item returns untyped, \texttt{void *}
	\item \texttt{NULL} if there's no more memory.
\end{itemize}

\begin{minted}{c}
#include <stdlib.h>

int main(void) {
	// array of 50 ints
	int *ip = (int *) malloc (50 * sizeof(int));
	
	// typecast is optional
	double *dp = malloc(1000 * sizeof(double));
}
\end{minted}

\subsection{\texttt{free()}}
\begin{minted}{c}
void free(void *p);
\end{minted}
\texttt{p} must be the same address returned by \texttt{malloc}! Don't \texttt{free} a pointer that's not from \texttt{malloc}!