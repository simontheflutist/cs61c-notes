\chapter{Operating Systems}
What is an operating system?
\begin{enumerate}
	\item First thing that runs when the computer starts.
	\item Find and control all devices.
	\item Starts services (filesystem, network, keyboard).
	\item Loads, runs, and manages programs:
	\begin{enumerate}
		\item Needs to run many programs at the same time (time-sharing).
		\item Needs to isolate programs from each other.
		\item Resources are multiplexed between applications.
	\end{enumerate}
\end{enumerate}

How does a program interact with the real world (devices)? Connections are maintained by \emph{buses}. (PCI is parallel; many individual device buses are serial.)

How does the processor do I/O?
\begin{enumerate}
	\item Special input/output instructions (e.g.~per device, tuned to hardware).
	\item Memory-mapped I/O: I/O devices are named by memory addresses; load/store using normal instructions. There are a lot of control (key down) and data (what key?) registers.
\end{enumerate}
Two ways to handle I/O:
\begin{description}
	\item[polling] wait for device to set \texttt{ready} bit, then read/write. (\texttt{ready} could be read or write control.)
	\item[interrupt] when I/O is ready or needs attention, control is sent to the \emph{interrupt handler}.
\end{description}
Traps are the software mechanism for dealing with unusual behavior.
\begin{description}
	\item[interrupts] are asynchronous signals (e.g.~I/O is ready) to enter the trap handler.
	\item[exceptions] are synchronous errors (e.g.~illegal memory R/W)
\end{description}
How is an interrupt handled?
\begin{enumerate}
	\item Instruction stream is suspended.
	\item CPU looks up the vector (function address) of interrupt handler in \emph{interrupt vector table} stored in CPU.
	\item \textsc{jal} to the handler, save PC in \textbf{M}achine \textbf{E}xception \textbf{P}rogram \textbf{Counter}.
	\item Handler runs, then returns:
\begin{minted}{asm}
handler: save registers
         decode interrupt
         handle interrupt
         clear interrupt
         restore registers
         mret
\end{minted}
\end{enumerate}
Exceptions are handled like pipeline hazards on a pipelined CPU:
\begin{enumerate}
	\item Complete instructions begun before exception.
	\item Flush instructions currently in pipeline.
	\item (optional) Store exception cause in status register.
\end{enumerate}
What happens at boot? (CPU heads first into a ROM.)
\begin{enumerate}
	\item BIOS finds a storage device and loads first sector.
	\item Bootloader loads kernel and jumps into it.
	\item OS initializes services and stuff.
	\item Launch and application that loops for user interaction.
\end{enumerate}
The OS runs in supervisor mode, in order to handle things like traps.
To read a file, launch a process, or allocate memory you need to perform a system call: set up registers and raise a software interrupt (which will be handled by OS).
All resources and devices are touched only by the OS.

How can multiple programs run simultaneously?
The OS just switches between processes very quickly (using a context switch),
using timed interrupts to allocate processor time.