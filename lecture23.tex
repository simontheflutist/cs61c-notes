\chapter{Virtual Memory}
Virtual memory helps enforce isolation: each program thinks it's running by itself.
Every thread has its own processor state, but they share memory.
Processes, however, will not share memory.
Virtual addresses, however neatly arranged, are mapped to physical addresses onto physical memory.
There is hardware indirection: two adjacent virtual addresses
can be very distant stored on physical memory (or even on the hard drive).

Responsibilities of memory manager:
\begin{enumerate}
	\item Map virtual to physical addresses.
	\item Protection: programs cannot touch other programs' memory or OS memory.
	\item Swap memory to disk in order to provide abstraction of larger memory. 
\end{enumerate}

\section{Paged memory}
The fundamental unit of memory is a \emph{page} of greater than 4\,KiB. (The slower the access, the bigger chunks you want to check out at a time.)
A 32-bit virtual address identifies a page number with its 20 upper bits and an offset with its 12 lower bits.

The memory manager has a \emph{page table} that subs out the 20 upper bits of a virtual address with a new 20 upper bits.
The OS can also assign the same physical page to several programs.
Pages can have write protection (e.g.~to prevent memory writes to the \emph{code} segment).

If you kept a full page table for 23-bit virtual addresses at 4\,KiB pages, that single table would be 4\,MiB (too large for a cache).
Instead, we will store page tables in DRAM. Now there are two more bits added to the page table rows:
\begin{description}
	\item[valid] whether the page is allocated.
	\item[DRAM/disk] tells where the page is stored (memory or disk).
\end{description}
As ``cache,'' should virtual memory be write-through or write-back?
At least code memory should always be clean.

\section{Storing page tables}
According to the previous calculation, you'd need 1\,GiB of memory just for all the page tables. Solution to big pages:
\begin{description}
	\item[increase page size] can potentially waste memory
	\item[hierarchical page tables] optimization insight: most programs use only top and bottom of memory.
\end{description}

\subsection{Hierarchical page table}
A processor register contains a pointer to the \emph{level 1} page table, in which each entry is a pointer to a \emph{level 2} page table (which maps onto physical memory). Unused sections of memory can correspond to unallocated L2 page tables.

The L1 page table must remain in memory; the L2 tables can be paged.

\section{Address translation}
Every instruction and data access requires looking up an address and checking bit flags. To make this happen in less than one cycle, use \emph{translation lookaside buffers}. A TLB is a fully associative cache \((\text{virtual page number})\to(\text{physical page number})\) that includes \emph{valid} and \emph{DRAM/disk} flags.