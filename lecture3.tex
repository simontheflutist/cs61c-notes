\chapter{Pointers}
\section{Pointers in C}
\begin{minted}{c}
int a;
a = -85;
printf("%d", a);
\end{minted}
\ldots yields the following (\emph{Type} and \emph{Name are just for notation}):

\noindent
\begin{tabular}{llrr}
	Type & Name & Addr & Value\\ \hline
	\vdots & \vdots & \vdots & \vdots \\
	\texttt{int} & \texttt{a} & 100 & -85 \\
	\vdots & \vdots & \vdots & \vdots \\
\end{tabular}

A pointer is a way of storing an memory address.

\begin{minted}[tabsize=4]{c}
int *x;     // x is an address of a int
int y = 9;  // y is an int
x = &y;     // x points to address of y
            // "reference" operator
int z = *x; // z gets the value of what x pointed to
            // "dereference" operator
*x = -7;    // assign -7 to x's target.
\end{minted}

\subsection{Generic pointer}
\begin{itemize}
	\item Points to \emph{any object}.
	
	\item \mint{c}{void *vp;}
\end{itemize}

\subsection{Pointer to struct}
\begin{minted}{c}
typedef struct { int x, y; } Point;

// initialize point object
Point pt = { 0, 5 };

// declare pointer
Point pt_ptr = &pt;

// access elements
(*pt_ptr).x = (*pt_ptr).y;

// but alternative syntax is cooler
pp->x = pp->y;
\end{minted}