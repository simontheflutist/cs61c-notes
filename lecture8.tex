\chapter{Running a Program—CALL}
\section{Instruction formats review}
Technically 6, but there are very nearly 4 formats.
\begin{description}
	\item[R-type] \verb|rd = rs1 OP rs2|
	\item[I-type] \verb|rd = rs1 OP imm|; load \texttt{rd} from memory at (\texttt{rs1} + \texttt{imm}); JALR.
	\item[S-type] Store \verb|rs2| to memory at (\texttt{rs1}+\texttt{imm}).
	\item[B-type] If (\texttt{rs1} cond \texttt{rs2}) branch to memory at (\texttt{PC} + \(2\left(\text{\texttt{imm}}\right)\)).
	\item[U-type] For huge numbers.
	\item[J-type] Used for JAL:
	\begin{itemize}
		\item \texttt{PC} + 4 goes into \texttt{rd}, 0 if a return is not necessary.
		\item \texttt{PC} = \texttt{{PC}} + \(2\left(\text{\texttt{imm}}\right)\)
			\begin{itemize}
				\item Can jump up to \(\pm2^{18}\) 32-bit instructions, \(\pm2^{20}\) bytes.
			\end{itemize}
	\end{itemize}
	also used for JALR:
	\begin{itemize}
		\item \texttt{JALR rd, rs, imm} writes (\texttt{PC} + 4) to \texttt{rd}.
		\item \texttt{ret} = \texttt{jalr x0, ra, 0}.
	\end{itemize}
	You can also use JALR to go to a 32-bit absolute address after \texttt{lui}.
\end{description}

\section{Multiply and divide}
\subsection{Integer case}
Integer multiplication is just like elementary school multiplication. When you multiply two 32-bit quantities, you could get a 64-bit value.
\begin{itemize}
	\item \texttt{MUL} gives you the lower 32 bits.
	\item \texttt{MULH} gives you the upper 32 bits.
\end{itemize}

Integer division is just like elementary school multiplication.
\begin{itemize}
	\item \texttt{div} stores floordiv.
	\item \texttt{rem} stores remainder.
\end{itemize}

\section{CALL}
\begin{description}
	\item[compiler] Input: high-level code. Output: assembly language, may contain pseudo-instructions.
	\item[assembler] Input: assembler files. Output: object code, which contains hints for the linker and the loader.
	\begin{itemize}
		\item Generates \emph{text} (code), \emph{data}, and \emph{info} sections.
	\end{itemize}
	\item[linker] Input: object code. Output: executable. \emph{Links} with resources. Four types of addresses:
	\begin{itemize}
		\item \texttt{PC}-relative addressing
			\begin{itemize}
				\item \texttt{beq}, \texttt{bne}, \texttt{jal}
				\item \texttt{lla}: \texttt{auipc}, \texttt{addi}
			\end{itemize}
		\item Absolute/external function addresses
			\begin{itemize}
				\item \texttt{la}: \texttt{auipc}/\texttt{lw} + \texttt{jalr}
			\end{itemize}
		\item Static data references
			\begin{itemize}
				\item \texttt{lui}/\texttt{addi}
			\end{itemize}
	\end{itemize}
	\item[loader] Input: executable code. Output: \emph{program gets run}.
		\begin{enumerate}
			\item Reads executable's header to determine size of text and data.
			\item Creates large enough address space and stack segment.
			\item Copies instructions and data into memory.
			\item Copies arguments passed onto the stack.
			\item Initializes registers.
			\item Jump to start-up routine that copies arguments into registers and sets the PC.
		\end{enumerate}
\end{description}
