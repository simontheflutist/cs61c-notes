\chapter{More RISC-V ISA}
\subsection{Magnitude comparisons}
\texttt{blt} ``branch less than'' is not the same as \texttt{bltu} ``branch less than unsigned.''

\subsection{Loop example}
\begin{multicols}{2}
	\begin{minted}{c}
int A[20];
int sum = 0;
for (int i = 0; i < 20; i++)


	sum += A[i]
	\end{minted}
	\columnbreak
	\begin{minted}{asm}
	addi x9, x8, 0    # x9 = &A[0]
	addi x10, x0, 0   # sum = 0
	addi x11, x0, 0   # i = 0
Loop:
	lw x12, 0(x9)     # x12 = A[i]
	add x10, x10, x12 # sum += x12
	
	addi x9, x9, 4    # &A[i++]
	addi x11, x11, 1  # i++
	addi x13, x0, 20  # x13 = 20
	blt x11, x13, Loop
	\end{minted}
\end{multicols}

\section{Programs}
\subsection{How a program is stored}
One RISC-V instruction is 32 bits. It's next to all the other data.
\subsection{Program execution}
\begin{itemize}
	\item PC is a register holding byte address of next instruction to be executed.
	\item Instruction fetched, control unit executes using datapath and memory system, increment PC; repeat.
\end{itemize}

\section{RISC-V assembler}
\subsection{Helpful features}
\begin{itemize}
	\item \texttt{a0}--\texttt{a7} correspond to \texttt{x10}--\texttt{x17}, \emph{argument registers}
	\item \texttt{zero} = \texttt{x0}
	\item \texttt{mv rd, rs} = \texttt{addi rd, rs, 0}
	\item \texttt{li rd, 13} = \texttt{addi rd, x0, 13}
\end{itemize}

\subsection{Function call}
It takes six steps to call a function.
\begin{enumerate}
	\item Put the parameters in a place where the function can access them.
	\item Transfer control to function.
	\item Acquire local storage for function.
	\item Perform function's desired task.
	\item Put result value in a place where calling code can access it; then release storage.
	\item Return control to caller.
\end{enumerate}

\subsection{Function call conventions}
\begin{itemize}
	\item \texttt{a0}--\texttt{a7} store arguments.
	\item \texttt{a0}--\texttt{a1} store return values.
	\item \texttt{ra}=\texttt{x1} holds return address (where you jump back to when you're done).
	\item\begin{minted}{c}
// somewhere
{sum(a,b);}

int sum(int x, int y) {
	return x + y;
}
	\end{minted}
	
	\begin{verbatim}
	1000 mv a0, s0 # x = a
	1004 mv a1, s1 # y = b
	1008 addi ra, zero, 1016 # stores ra
	1012 j sum
	1016 # return here
	# ...
	2000 sum: add a0, a0, a1 # does subroutine logic
	2004 jr ra # return control
	\end{verbatim}
	\item \texttt{jal} (jump and link) writes to \texttt{ra} and jumps. (shortcut for \texttt{addi} and \texttt{j})
	\item \texttt{ret}  = \texttt{jr ra}
\end{itemize}

\subsection{Function call example}
\begin{multicols}{2}
\begin{minted}{c}
// C version
int Leaf
(int g, int h, int i, int j) {
	int f;
	
	
	
	f = (g + h) - (i + j);
	
	
	
	
	
	
	
	return f
}
\end{minted} 
\columnbreak
\begin{verbatim}
# adjust stack for two items
Leaf: addi sp, sp, -8

# push s0 and s1 onto the stack
      sw s1, 4(sp)  
      sw s0, 0(sp)
	  
      add s0, a0, a1
      add s1, a2, a3
      sub a0, s0, s1
	  
# pop s0 and s1]	
      lw s0, 0(sp)
      lw s1, 4(s0)
      addi sp, sp, 8
      jr ra
\end{verbatim}
\end{multicols}

Old registers are stored in the stack. The \emph{stack pointer} is \texttt{sp}=\texttt{x2}.

\subsection{Function that calls a function}
\texttt{a0}--\texttt{a7} and \texttt{ra} can't be overwritten!
\begin{minted}{c}
int sumSquare(int x, int y) {
    return mult(x, x) + y;
}
\end{minted}

RISC-V divides registers into two categories:
\begin{enumerate}
	\item Preserved across function call (you have to push and restore):
		\begin{itemize}
			\item \texttt{sp} \emph{stack pointer}
			\item \texttt{gp}
			\item \texttt{s0}--\texttt{s11} \emph{saved registers}
		\end{itemize}
	\item Not preserved across function call (free to use). 
		\begin{itemize}
			\item \texttt{a0}--\texttt{a7} \emph{argument/return registers}
			\item \texttt{ra} \emph{return address}
			\item \texttt{t0}--\texttt{t6} \emph{temporary registers}
		\end{itemize}
\end{enumerate}

Whereas C has two storage classes.
\begin{description}
	\item[automatic] variables local to function, discarded
	\item[static] variables that persist globally
\end{description}