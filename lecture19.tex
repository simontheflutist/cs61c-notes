\chapter{Thread-level parallelism}
A \emph{core} is an independent processor, with control and datapath (PC, registers, and ALU).
Shared resources include memory and often L3 cache.

A \emph{thread} is a sequential flow of instructions that performs some task (``program'').
Each thread has PC and registers and shared memory. A physical core provides \emph{hardware threads} that execute simultaneously.
The OS multiplexes \emph{software threads} onto \emph{hardware threads} (those not running are sleeping).

\subsection{Hardware-assisted software multithreading}
In one core with two threads, some datapath elements (ALU) are shared, while some state elements (PC and registers) are separate.

\begin{itemize}
	\item Logical threads:
		\begin{itemize}
			\item 1\% more hardware, 10\% better performance
		\end{itemize}
	\item Multicore:
		\begin{itemize}
			\item 50\% more hardware, 100\% better performance
		\end{itemize}
\end{itemize}

\section{OpenMP}
\subsection{\texttt{for} loops}
\begin{itemize}
	\item Serial:
\begin{minted}{c}
for (int i = 0; i < 100; i++) {
	// stuff
}
\end{minted}
	\item Parallel:
\begin{minted}{c}
#include <omp.h>

#pragma omp parallel for
for (int i = 0; i < 100; i++) { /* ... */ }
\end{minted}
\end{itemize}
OpenMP uses a fork-join model (just like Java's \texttt{Stream.parallel()}).