\chapter{DMA, Disks, and Networking}
\subsection{Virtual memory in the RISC-V pipeline}
The TLB stands in the way of instruction and data access.
It is only 32--128 entries, fully associative for maximum speed.
Page faults are exceptions that hand execution to a different thread.

In memory-mapped I/O, we learned ``programmed I/O'' in which the CPU is instructed to do I/O via memory addresses and I/O is buffered through memory when \texttt{lw}/\texttt{sw} instructions are given.
In \emph{direct memory access}, we will develop an independent FSM
that handles data transfer.

\section{Direct Memory Access}
Problems with programmed I/O:
\begin{enumerate}
	\item CPU should be doing work instead of cycling through transfers.
	\item CPU is way faster than devices.
	\item CPUs use a lot of power when simpler hardware could suffice.
\end{enumerate}
But ``programmed I/O'' also gives up the invariant we've maintained until now that the CPU should have sole control over main memory.
With DMA, we will remove the CPU middleman and allow I/O devices to target memory directly.

The \emph{DMA engine} will take registers written by the CPU:
\begin{enumerate}
	\item memory address to write to
	\item number of bytes
	\item which device to interact with (controllers can handle many devices) and direction of transfer
	\item unit of transfer
\end{enumerate}

\subsection{Device \(\to\) memory}
\begin{enumerate}
	\item CPU programs DMA controller.
	\item DMA request transfer from device.
	\item Device transfers to memory.
	\item Device signals DMA.
	\item DMA fires interrupt.
\end{enumerate}

\subsection{Memory \(\to\) device}
\begin{enumerate}
	\item CPU confirms device is ready.
	\item CPU directs DMA to start transfer.
	\item DMA does the transfer (CPU checks out).
	\item DMA sends interrupt.
\end{enumerate}

Where in the memory hierarchy should the DMA go?
\begin{itemize}
	\item between L1 cache and CPU:
	\begin{description}
		\item[Pro] free coherency
		\item[Con] CPU's working set filled with alien data; ruins spatial/temporal locality in the short term (which is what L1 is for)
	\end{description}
	\item between last cache and main memory:
	\begin{description}
		\item[Pro] no cache interference
		\item[Con] more difficult for coherency---can invalidate caches
	\end{description}
\end{itemize}

\section{Disk drive performance}
Disk Access time is sum of:
\begin{description}
	\item[seek time] time to position the head assembly on the proper cylinder
	\begin{itemize}
		\item on average, you have to move the arm over \(n_\text{tracks}/3\) tracks
	\end{itemize}
	\item[rotation time] time for the disk to rotate to where the first sectors of the block reach the head
	\begin{itemize}
		\item averages to half the time of a rotation (worst case)---on 7.2k\,rpm, this works out to 4.17\,ms.
	\end{itemize}
	\item[transfer time] time for the sectors of the block and gaps to rotate past the head
	\item[controller overhead] time for electronics to return data
\end{description}

\section{Networking}
Two kinds of networks:
\begin{description}
	\item[shared] one channel for everyone: wait for channel to become free before sending
	\item[switched] communication takes place between pairs only
\end{description}
Aggregate bandwidth in a switched network is much bigger.

Network payload contents:
\begin{description}
	\item[header] with destination, source, and length
	\item[payload] with data
	\item[trailer] with a checksum
\end{description}

Steps to send:
\begin{enumerate}
	\item App copies into OS buffer.
	\item OS builds packet and starts timer.
	\item OS sends data to network interface and says to start.
\end{enumerate}
Steps to receive:
\begin{enumerate}
	\item OS copies data from interface into buffer.
	\item OS checks checksum: ACK if okay, otherwise ignore (sender will resend on timer).
	\item Copy data to memory map, interrupt to resume.
\end{enumerate}

Networking takes layers:
\begin{enumerate}
	\item Application (chat, game, etc.)
	\item Transport (TCP, UDP)
	\item Network (IP)---making the connection
	\item Data link layer (ethernet)---handshakes to share physical resources
	\item Physical link (shape of physical signal)
\end{enumerate}
Every layer encapsulates information from above by adding a header and a trailer. IP tries to deliver; TCP guarantees delivery.