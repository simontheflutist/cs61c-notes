\chapter{RISC-V Instruction Encoding}
Computers nowadays are the stored-program sort.
This design has consequences:
\begin{enumerate}
  \item Everything is a memory address.
    Instructions and data live side by side, and it is easy to break things.
  \item Binary compatibility must be a thing.
    People want old programs to run on new systems. Most ISAs are backwards-compatible.
    \begin{itemize}
      \item Most data is words---how then, are instructions represented?
      \item Every line of assembly is mapped to a 32-bit word.
        Instructions are the same size irrespective of word size!
      \item An instruction is broken up into fields according to type.
        \begin{description}
          \item[R-format] does register-register arithmetic.
          \item[I-format] handles instructions with immediates. S-format allows only 5 bits, but that might not be enough. In the I-format, the funct7 code becomes part of a 12-bit immediate. At runtime, this immediate is sign-extended to 32 bits.
	          \begin{itemize}
	          	\item Shift-immediate instructions are an exception and use only 5 bits since you can't shift more than 31. (They are the LSBs of the standard I-format.)
	          	\item Load instructions are also I-type! The ``immediate'' field encodes the offset.
	          \end{itemize}
          \item[S-format] is for store instructions. Different from I-format Load because the immediate field is split into \(8+4\).
          
	      \begin{quotation}
	      	\textsc{Warning} \texttt{rs1} and \texttt{rs2} seem to be reversed!
	      \end{quotation}
          \item[B-format] is for branch instructions. The immediate is twos-complement PC offset that is multiplied by 2 to increment the PC, and the immediate is broken up all over the place. (You can represent -4096 to 4094 PC displacement.)

          Example:
          \begin{minted}{asm}
Loop:  beq x19, x10, End
       add x18, x18, x10
       addi x19, x19, -1
       j Loop
End:   # target
          \end{minted}
          In this loop, the conditional branch to \texttt{End} is 4 instructions, which is \(4\cdot32\) \text{bits}, which is \(16 = 8\cdot2\) bytes. The immediate field of this branch code
          \begin{itemize}
          	\item Why is immediate encoding so painfully irregular? It's to simplify the hardware that aligns immediate encoding to an immediate value.
          \end{itemize}
          \item[U-format] for upper immediate. This one is weird because it has a huge 20-bit immediate and a destination and nothing else. If you want an entire 32-bit constant, first you \texttt{lui} (load upper immediate) the top 20 bits than \texttt{addi} the last 12.
          \begin{itemize}
          	\item There is a horrible complication! \texttt{addi} is always sign extended, so if the 12th bit is 1, \texttt{addi} will subtract 1 from the upper 20 bits.
          	\item This is a horror that is hidden by a gracious assembler pseudo-op: \texttt{li} loads a 32-bit constant that loads +1, \texttt{addi}s.
          \end{itemize}
          \texttt{auipc} adds a 20-bit int to the program counter.
        \end{description}
    \end{itemize}
\end{enumerate}
